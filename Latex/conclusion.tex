
\chapter{Conclusion}
This summer internship project successfully detailed the methodologies for the expression and purification of three significant scorpion toxins: BmK M1, Lqq4, and the Lqh$\alpha$IT mutant (Y11S). The comprehensive approach, encompassing molecular cloning, bacterial expression, and multi-stage protein purification, provided hands-on experience in advanced protein biochemistry techniques.

The project commenced with the successful transformation of E. coli BL21(DE3) cells with pET21a(+) plasmids containing the respective toxin genes. Subsequent IPTG induction facilitated the expression of these toxins as fusion proteins. Cell lysis through the French press method effectively released the intracellular proteins, preparing them for downstream processing.

The purification involved a sequential application of established chromatographic techniques. Ni-NTA affinity chromatography proved effective in isolating the His-tagged fusion proteins, as evidenced by SDS-PAGE analysis showing concentrated bands in the elution fractions and minimal protein in the flow-through and wash steps. Following affinity purification, thrombin cleavage successfully liberated the target peptides from their fusion partners. Size exclusion chromatography (SEC) further refined the purification by separating the peptides effectively removing larger contaminants such as the cleaved Cytb$_{5}$ and His-tag. Finally, reverse-phase HPLC (RP-HPLC) provided the high-resolution separation necessary to achieve the desired purity, with distinct peaks corresponding to the target peptides. The successful purification of unlabelled Lqq4 was confirmed by MALDI mass spectrometry, which showed a mass consistent with the expected molecular weight.

This internship has been a profoundly formative experience, providing a deep understanding of practical protein biochemistry methodologies, expression system optimization, and the biochemical behavior of disulfide-rich peptides. The challenges encountered during the various stages of purification, such as optimizing buffer conditions and troubleshooting chromatographic runs, offered critical problem-solving opportunities. The insights gained into the expression and purification of these specific scorpion toxins, particularly their unique structural features like the non-proline cis peptide bond, contribute to a broader understanding of their functional roles and potential therapeutic applications. This experience has built a strong foundation for future research endeavors in protein science and structural biology.

\chapter{Future Scope}
\begin{itemize}

\item Verification of successful purification of remaining peptides using MALDI MS

\item Expression and purification of isotope labelled peptides

\item Analysing the unlabelled and labelled peptides using NMR
\end{itemize}
