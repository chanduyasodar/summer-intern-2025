\documentclass[12pt,a4paper]{report}
\usepackage{graphicx}
\usepackage{titlesec}
\usepackage[export]{adjustbox}
\usepackage{hyperref}
\usepackage{gensymb}
\usepackage{tabularx}
\usepackage{float}
\usepackage{array}
\usepackage{wrapfig}
\usepackage{subcaption}
\usepackage[backend=biber]{biblatex}
\addbibresource{bibliography.bib}
\usepackage{amsmath}
\usepackage{amssymb}
\usepackage{caption}
\usepackage{tikz}
\usetikzlibrary{shapes.geometric, arrows}
\usepackage[dvipsnames]{xcolor}
\usepackage{blindtext}
\usepackage[a4paper, total={6in,9.5in}]{geometry}

\begin{document}
\begin{titlepage}

\begin{center}
\vspace{1in}
\textup{\small {\bf Summer Internship Project} \\ Report}\\[0.3in]

% Title
\Large \textbf {Expression and purification of Scorpion toxins\\Bmk M1, Lqq4 and Lqh$\alpha$IT Mutant}\\[0.7in]


       

% Submitted by
\normalsize Submitted by \\[0.2in]
\textbf{M.R.Chandu Yasodar}\\
BSMS student\\
IISER Tirupati\\

\vspace{.2in}
Under the guidance of\\[0.2in]
\textbf{Prof. Siddhartha P. Sarma}\\




\vspace{.3in}

% Bottom of the page
\includegraphics[width=0.2\textwidth]{pics/iisc logo.jpg}\\[0.1in]
\Large{Molecular Biophysics Unit  }\\
\normalsize
\textsc{Indian Institute of Science(IISc)}\\
Bengaluru - 560012 \\
\vspace{0.2cm}
Summer Internship 2025

\end{center}

\end{titlepage}




\phantomsection
\addcontentsline{toc}{chapter}{Acknowledgement}
\chapter*{Acknowledgment}
I would like to express my deepest gratitude to \textbf{Prof. Siddhartha P. Sarma}, for giving me the opportunity to work at the lab in Indian Institute of Science(IISc). Being part of this research environment and access to lab resources truly helped me build a strong foundation in protein expression and purification.

I am sincerely thankful to my mentor, \textbf{Suvedhaa}, for the many conversations, clarifications, and encouragement that helped me grow both technically and personally. Your patience and willingness to share your knowledge played a key role in my learning experience.

A special thanks to the lab members - \textbf{Sri Teja, Kamesh \& Anjali} for welcoming me into the team and for creating a collaborative and friendly environment. Your readiness to help, thoughtful discussions, and camaraderie made this internship enjoyable and unforgettable.

This internship has been a formative experience, and I feel fortunate to have worked with such a dedicated and passionate group.


\tableofcontents
\newpage
\listoffigures
\newpage
\listoftables



\phantomsection
\addcontentsline{toc}{chapter}{Abstract}
\chapter*{Abstract}
This report documents the expression and purification of BmK M1, originally derived from \textit{Buthus martensii Karsch}, Lqq4, originally derived from \textit{Leiurus quinquestriatus quinquestriatus}(Death stalker scorpion), Lqh$\alpha$IT mutant(Y11S), whose wild type is originally derived from \textit{Leiurus quinquestriatus hebraeus}. The project was conducted under the supervision of Prof. Siddhartha P. Sarma, with the primary objective of gaining hands-on experience in protein expression and purification. Respective toxin genes were cloned in pET21a(+) plasmid, transformed into E.coli BL21(DE3) strain and were expressed using IPTG induction. The cells were lysed and the peptides were purified through Ni-NTA Affinity chromatography, subsequently cleaved by thrombin to release the peptide of interest from the fusion protein. The cleaved products were concentrated and purified using size exclusion chromatography and reverse phase HPLC. The purified products were lyophilised. This internship provided a deep understanding of practical protein biochemistry methodologies, expression system optimisation and biochemical behaviour of disulfide-rich peptides such as the above-mentioned scorpion toxins. 







    \chapter{Introduction}
Scorpions are one of the most ancient group of animals on Earth(more than 400 millions of years of evolution), represented now by approximately 1500 different species, which have conserved their morphology almost unaltered. They belong
to the class \textit{Arachnida}, order \textit{Scorpionida}. Their toxins have played a very crucial role in their evolutionary success  by aiding in survival, defence and predation. Their venom is a complex mixture of various of biologically active components: enzymes, peptides, nucleotides, lipids, mucoproteins, biogenic amines and other unknown substances\cite{scorpion_toxin}. The known number of different components in their venoms varies from 72 to over 600.\\

The properties of scorpion toxins that attract the researchers is their specificity regarding their activity on certain phyla and their site of attack. They are very specific about the phyla the toxins would act on, i.e, a certain group of scorpion toxins are specific to crustaceans and a few are specific to mammals alone. Another such feature is their site of action. They modify the activation \& inactivation kinetics of voltage-gated channels and ion-channel permeability of both excitable and non-excitable cells. Any information obtained about these can be applied in the biomedical, pharmacology and immunology fields\cite{scorpion_toxin}.\\

 In this project, I worked particularly Lqh$\alpha$IT mutant(tyrosine in 11th position is replaced with serine) which stands for \texti{Leiurus quinquestriatus hebraeus} alpha-insect toxin, Lqq4, extracted from venom glands of \textit{Leiurus quinquestriatus quinquestriatus} (aka Death Stalker) and BmK M1, isolated from \textit{Buthus martensii Karsch} (aka Manchurian scorpion). These 3 toxins affect the Na\textsuperscript{+} gated voltage channels. These 3 toxins also have one very special common feature, that is, presence of non-proline cis peptide bond  in their core. 

Let's dive deeper into their structure, mechanisms, expression, purification and it's chracterisation.

\chapter{Review of literature}
\section{Classification of Scorpion toxins}
The General classification of scorpion toxins is based on four different criteria: the ion-channel involved (sodium, potassium, calcium and chlorine), the specific receptor to which the toxin binds to, the three-dimensional structure of the toxin and the type of response induced (activation/inactivation of the receptor)\cite{scorpiontoxinclass}. We will focus of Na\textsuperscript{+} binding toxins.

\subsection{Na\textsuperscript{+} channel specifc toxin}
These consist of 61 to 76 amino acid residues, highly packed with 4 disulfide bonds. They are subclassified based on the receptor they bind to and their physiological effects: alpha-toxins ($\alpha$-
NaScTxs), which bind at receptor site 3 on the extracellular surface of the channel and inhibit the fast inactivation process, and beta-toxins ($\beta$-
NaScTxs), which bind to receptor site 4 and shift the
threshold of the channel activation to more negative
membrane potentials

\subsection{$\alpha$-NaSctx toxins}
The a-NaScTxs are sub-divided into distinct groups 
\begin{itemize}
    
\item  Classical $\alpha$-toxins, are highly active only in mammalian voltage-gated sodium channels (VGSCs) with high affinity to rat brain synaptosomes. Among these toxins are: Aah2, Aah1 and Aah3 from Androctonus australis Hector,  and Bot3 from Buthus occitanus tunetatus, peptides purified from North African scorpions 

\item Anti-insect $\alpha$-NaScTXs, that are highly active only on insect VGSCs. Examples of these toxins are:\textbf{ Lqh$\alpha$IT} from Leiurus quinquestriatus quinquestriatus, \textbf{Lqq4} from Leiurus quinquestriatus quinquestriatus, Lqq3 , and BotIT1 which bind with high affinity to insect neuronal preparations.

\item (3) $\alpha$-Like toxins, active on both insect and mammalian VGSCs. Examples are: Lqh3 and Lqh6 (from L. quinquestriatus hebraeus), Bom3 and Bom4 (from italicize it if it is a scienctific name), and \textbf{BmK M1} from Buthus martensii Karsch.
\end{itemize}




\section{3D Structure of $\alpha$-NaSctx toxins }

\begin{wrapfigure}{H}{0.45\linewidth}
\centering
     \includegraphics[height=2.5\linewidth,]{pics/sequence.png}
    \caption{Structure of the BmK M1, Lqq4 and Lqh$\alpha$IT mutant, respectively }
    \label{fig:enter-label}
\end{wrapfigure}

The core structure contains six cysteines which form three conserved disulfide bridges and a fourth disulfide bridge can be formed in three different arrangements. These toxins have an essential three-dimensional structure highly conserved, comprising an $\alpha$-helix and three or four-stranded anti-parallel $\beta$-sheets stabilized by four spatially conserved disulfide bridges. The helix motif is linked to $\beta$-3 strand by two of the four disulfide bonds. The cysteine pair of the $\alpha$-helix motif is spaced by a tripeptide, whereas the pair of cysteine residues of the $\beta$-3 strand is separated by only one amino acid residue. By numbering the cysteine residues progressively from the N- to C-terminus, the $\alpha$-helix would contain the residues Cys3 and Cys4 and the $\beta$3 strand would include residues Cys6 and Cys7, forming two conserved disulfide bridges: Cys3–Cys6 and Cys4–Cys7. A third structurally conserved disulfide bond is the one between the Cys residue located at the $\beta$-2 strand, Cys5, and Cys2, except for the excitatory insect toxins. The three first conserved disulfide bridges are involved in the stabilization of a structurally conserved core of scorpion toxins. The fourth disulfide bridge in the majority of toxins is established between the most N- and C-terminal cysteine residues (Cys1 and Cys8). In figure 2.1, yellow coded sequence stands for $\beta$-sheets, orange highlighted sequece stands of $\alpha$-helix, green lines represents disulfide bonds and blue box represents non-proline cis peptide bond. \\


\subsection{Functional surface of $\alpha$-NaSctx toxins}
A functional surface called “NC-domain” comprises the five-residue turn, that interlaces with the N-terminal segment(residues 8–12) and the C-terminus (residues 56–64) and a core-domain formed by several residues (positively-charged and hydrophobic amino acids at the short loops connecting the conserved secondary structure elements of the molecule core) spatially in close proximity to the residue at position 18.

Our particular interest in this NC domain with respect to the peptides we are studying is that that NC domain comprises a non-proline cis peptide bond in them.

\section{Cis peptide bonds}
The partial double bond character of peptide bonds results in two conformations based on the dihedral angle i.e, $\omega$=0\degree refers cis and $\omega$=180\degree refers to trans isomer. Cis isomer is unstable due to steric hindrance  between the $\alpha$ carbons of the bonded amino acids, so, trans conformer is mostly preferred by the peptide. On the basis of the potential energy difference and the Boltzmann distribution, the probability of occurrence of cis peptide bonds has been estimated to be between 1\% and 1.5\%\cite{cis}. A difference in energy of approximately 2.5 kcal/mol is observed between the trans and the cis isomers (corresponding to only 1.5 \% occurrence of the cis form), regardless of the solvent, and a rotational barrier of about 20 kcal/mol have been found for the peptide bond analog N-methylacetamide. For an imide bond in Pro-containing peptides, however, the trans isomer is favoured over the cis by only 0.5 kcal/mol 1970), so that a higher abundance (10-30 \%) of the cis form is observed. The pyrrolidine ring of Pro can be associated with two types of puckering, designated UP and DOWN, depending on the ring torsion angles. Puckering of the ring when it is involved in the cis linkage is DOWN\cite{cis2}.

\subsection{Non-Proline cis peptide bond}
The occurrence of non-Pro cis peptide bonds has
been associated with steric strain in proteins. The two possible reasons for the occurrence of preferential stabilization are geometry and function.
The characterisation of the following peptides might lead us to insights about the previously discussion.






. \chapter{Materials and Methods}
\section{Strains and plasmids used}
For amplification of plamsid, \textit{E.coli} DH5$\alpha$ strain was used whereas for protein expression, BL21(DE3) was used.\\
Plasmid used is pET21a(+)\\
\begin{figure}[H]
    \includegraphics[width=1\textwidth]{pics/pET-21a(+) Map.png}
    \caption{Sequence Map of pET21a(+) \cite{map}}
    \label{fig:enter-label}
\end{figure}

\tikzstyle{selectionmarker} = [rectangle, rounded corners, 
minimum width=2cm, 
minimum height=1cm,
text width=2cm,
text centered, 
draw=black, 
fill=red!30]

\tikzstyle{io} = [trapezium, 
trapezium stretches=true,
trapezium left angle=70, 
trapezium right angle=110,
text width=2cm,
minimum width=2cm, 
minimum height=1cm, text centered, 
draw=black, fill=blue!30]

\tikzstyle{process} = [rectangle, 
minimum width=2cm, 
minimum height=1cm, 
text centered, 
text width=2cm, 
draw=black, 
fill=orange!30]

\tikzstyle{decision} = [diamond, 
minimum width=2cm, 
minimum height=1cm, 
text centered,
text width=2cm,
draw=black, 
fill=green!30]
\tikzstyle{arrow} = [thick,->,>=stealth]
\begin{figure}
    \centering
    
    
\begin{tikzpicture}[node distance=2cm]

\node (sm1) [selectionmarker] {Nde I};
\node (in1) [io, right of=sm1, xshift=0.5cm] {6-Hisidine Tag};
\node (pro1) [process, right of=in1, xshift=0.5cm] {Cytochrome

b\textsubscript{5}};
\node (dec1) [io, right of=pro1,xshift=0.5cm] {Thrombin

Cleavage

Site};

\node (pro2a) [decision, right of=dec1, xshift=0.7cm] {Peptide

Sequence};

\node (pro2b) [selectionmarker, right of=pro2a, xshift=0.5cm] {EcoR I};



\end{tikzpicture}
\caption{Sequence cloned into the plasmid}
    \label{fig:enter-label}
\end{figure}


\subsection{Rationale behind using Cytb\textsubscript{5} and 6-Histidine tag}
Recombinant peptides in general, don't express in sufficient amounts in the host as they aren't expressed by the host naturally. They often pertain to express in low levels or forms insoluble inclusion bodies. So, in order to overcome this, the peptide of interest is expressed with a fusion tag. The peptide is attached to a host or parent protein, expressed as a single molecule and cleaved out later. Parent protein that is chosen is one that expresses well in bacterial cultures, highly soluble, stable, and easy to purify. One such parent protein is heme binding domain of rat cytochrome b\textsubscript{5}. It expresses in extremely high levels in common E. coli host expression strains and is very soluble, and thus satisfies all the criteria requisite of a good fusion protein host\cite{cytb5}.

6-histidine tag is added for affinity chromatography and it's mechanism is explained further in 3.3.2 .


\section{Expression}
The text in \textit{italics} suggests that the respective step was done in a laminar hood under sterile conditions
\subsection{Transformation}
First, LB agar plates were prepared. LB Ag solution(25g LB, 17g agar per L of water) was autoclaved at 121 °C for 30 minutes.\textit{ Antibiotic (ampicillin in this case) was added, final concentration = 100 \textmu g/mL. Then,  media was poured into petri plates before the solution solidifies }, and the plates were stored in 44\degree C

The method used here is \textbf{Heat Shock Method}\cite{heat_shock}. The competent cells and plasmid were taken out and kept on ice. \textit{In a 1 mL Eppendorf tube, to 100 \textmu L of the competent cells,  2 \textmu L of plasmid stock solution was added and was placed on ice for 10 minutes. Then, tube was placed in 42 \degree C for 90 seconds and back in ice for 5 minutes. 1 mL of  LB Media to the tube was added} and was incubated in 37\degree C and 180 rpm. The tube was centrifuged at 6000 rpm for 3 minutes. \textit{800 \textmu L of media was removed without disturbing the pellet, then, the pellet was resuspended in the remaining 200 \textmu L, which was spread plated in the LB agar plates.} The plates were incubated at 37\degree C for 12 hours or overnight. If the transformation was successful, we can observe opaque, white dots after incubation. After incubation, the transformed culture was stored at 4\degree °C. 

\subsection{Cell Culture}
For primary culture, regardless of the labelling of the peptide, LB media is used. 25 mL LB media was prepared in a 100 mL conical flask and autoclaved at 121\degree C for 30 minutes. \textit{Antibiotic was added(ampicillin,100 \textmu g/mL). Few transformed colonies were scraped using a pipette tip and the tip was dropped into the primary media} , then, incubated in 37\degree C for 12 hours or overnight. If precipitate is observed, it means that the media has dead cells.

For secondary culture, if isotopic labelling is not going to be performed, then LB media can be used. 500 mL of LB media was prepared in a 2 L conical flask, and two such flasks were made, so a total of 1 L of LB media was made and autoclaved.\textit{Antibiotic was added(ampicillin,100 \textmu g/mL)}.

\subsubsection{M9 Minimal media}
This kind of media is prepared if we have to isotopically label the protein with $^{15}N$ or $^{13}C$ or both of thetwo wo 2   conical flasks, a 500mL Schott Duran bottle with 300 mL MillQ water, a 1 L Schott Duran bottle with 700 mL filtered water using a 0.2 \textmu filter paper, along with the filtering unit, were autoclaved. Measure the following compounds according to the isotope labelling done. If the protein is being $^{15}N$ labeled,$^{15}NH_4Cl$ will be used. If the protein is being $^{13}C$ labeled, $^{13}C_6H_{12}O_6$ shall be used. If the protein is to be double-labelled, both of the above compounds will be used.

\begin{wraptable}{b}{6cm}

\begin{tabularx}{6cm} { 
  | >{\centering\arraybackslash}X  
  | >{\centering\arraybackslash}X |}
  \hline
  $KH_2PO_4$ & 3g  \\ 
  \hline
  $Na_2HPO_4$ & 6g  \\ 
  \hline
  NaCl & 0.53 g \\ 
  \hline
  $NH_4Cl$ & 1 g  \\ 
  \hline
  Glucose &  2 g  \\ 
  \hline
  Thiamine & 1 mL  \\ 
  \hline
  Trace Elements &  10 \textmu L \\ 
  \hline
  100x Divalent cations & 10 mL  \\ 
  \hline
\end{tabularx}
\caption{Composition of 1 L M9 Minimal Media}
\label{table:ta1}
\end{wraptable} 

The compounds listed in table 3.1 were added to 300 mL of water, then \textit{it was filtered through the filtering unit attached to a Schott Duran bottle with 700 mL of water. Antibiotic (ampicillin, 100 \textmu g/mL) was added and mixed well. The medium was divided into two 2 L conical flasks} \\

\textit{After the secondary media is ready, 1 mL of media was pipetted into an Eppendorf so that it can be used as a blank for measuring OD$_{600}$. 1\%\ of post incubation primary culture was added to the secondary media i.e, 5 mL of primary culture to to each flask} and was incubated in 37\degree C in 180 rpm until it reached an OD$_{600}$ of 0.4 or 0.6(3 to 4 hours). 

Then, culture was induced with IPTG. \textit{Appropriate amount of IPTG was added such that its final concentration is 4 mM} and was incubated at 20\degree C for 16 hours. IPTG, isopropyl $\beta$-D-thiogalactopyranoside, is a molecular analogue of allolactose, which removes a repressor from the lac operon to induce gene expression\cite{iptg}. We can observe frothing in the culture. Same as  in the primary media, precipitation is not supposed to happen, if it is observed, it means that cells are dead. 

Post induction and incubation, it was centrifuged at 5500 rpm, 4\degree C for 20 minutes of the second culture to pellet the cells. The supernatant was discarded and pellets were stored at -80\degree C until the next step is performed



\section{Purification}

\subsection{Cell Lysis}

Cell lysis or cellular disruption is a method in which the outer boundary or cell membrane is broken down or destroyed in order to release intercellular materials such as DNA, RNA, protein or organelles from a cell. Cell lysis is an essential unit operation for molecular diagnostics of pathogens, immuno-assays for point of care diagnostics, down streaming processes such as protein purification for studying protein function and structure, cancer diagnostics, drug screening, mRNA transcriptome determination and analysis of the composition of specific proteins, lipids, and nucleic acids individually or as complexes.
\begin{wrapfigure}{t}{0.7\linewidth}
    \includegraphics[width=0.8\linewidth]{pics/cell_lysis.jpg}
    \caption{Classification of different methods of cell lysis }
    \label{fig:enter-label}
\end{wrapfigure}

Based on the application, cell lysis can be classified as complete or partial. Partial cell lysis is performed in techniques such as patch clamping, which is used for drug testing and studying intracellular ionic currents. Complete cell lysis is the full disintegration of cell membrane for analyzing DNA, RNA and subcellular components.\cite{cell_lysis}

\subsubsection{French Press}
\paragraph{Principle} French press comes under the mechanical method of cell lysis, specifically high pressure homogenizer. French press lyses the cell through disrupting it's plasma membrane and cell wall using the process of liquid homogenization where a external pump pushes a piston within a large cylinder containing the sample , forcing the sample through a minute hole. Thus the sample experiences a high shear force and decompression breaking the cell wall and plasma membrane.\\
\begin{wraptable}{l}{6cm}
\caption{Lysis Buffer}
\begin{tabularx}{5.7cm}{ 
  | >{\centering\arraybackslash}X 
  | >{\centering\arraybackslash}X
  | }
\hline
 Tris & 50 mM  \\
 \hline
 NaCl  & 300 mM    \\
\hline
\multicolumn{2}{|c|}{pH=8, using HCl}    \\
\hline
\multicolumn{2}{|c|}{Filter with 0.45\textmu\ filter paper}  \\
\hline
\end{tabularx}
\label{table:ta2}
\end{wraptable}

\paragraph{Procedure}Pellet was taken from the -80\degree C, thawed in ice and resuspended in 60 mL Lysis buffer. Serine protease inhibitor\cite{pmsf} was prepared by dissolving PMSF(phenylmethylsulfonyl fluoride) in ethanol(final concentration in the sample = 100 \textmu M). PMSF solution  has to be added frequently as it degrades within an hour. The French press apparatus was kept in ice for 15 to 20 minutes. Then, sample was loaded in the setup and processed in the machine. The cycle was repeated for 2-4 times depending on the outlet, if it's watery, then the cells have been lysed, else the process is repeated.

\begin{figure}[H]

\begin{subfigure}{0.5\textwidth}
\includegraphics[width=0.9\linewidth, height=6cm]{pics/fp1.jpg} 
\caption{Machine as a whole}
\label{fig:subim1}
\end{subfigure}
\begin{subfigure}{0.5\textwidth}
\includegraphics[width=0.9\linewidth, height=6cm]{pics/fp2.jpg}
\caption{Piston, needle valve, cylinder and the flow-through tube}
\label{fig:subim2}
\end{subfigure}

\caption{French press}
\label{fig:image2}
\end{figure}

Then, the flow-through was centrifuged at 12500 rpm, 45 minutes at 4\degree C. The cell debris is present in the pellet and the supernatant has the protein. 
\subsection{Affinity Chromatography}
  Property of target molecule's(mixed in mobile phase) specific binding with stationary phase i.e, a type of ligand is used to purify the molecule. A subcategory of affinity chromatography called \textbf{Immobilsed metal affinity chromatography} encompasses the property of affinity of amino acids like His or Cys to transition metal ions like Zn$^{2+}$, Cu$^{2+}$,Co$^{3+}$, Ni$^{2+}$ etc. A chelating ligand like NTA or IDA is used to fix the metal ion to agarose.\cite{niNTA,niNTA2}

\subsubsection{Ni - NTA Column}\
\paragraph{Principle}Histidine has a imidazole ring in it's side chain, and it has high affinity for the transition metal ion Ni$^{2+}$. The protein will have His in it's sequence but, for ensure strong and specific binding, a poly Histidine tag is added to the sequence to the protein we have to purify. This is done by modifying the genetic sequence artificially. In this case, the poly histidine tag has 6-histidines. Ni$^{2+}$ has a coordination number of 6, NTA occupies 4 of Ni$^{2+}$, 2 of the 6-histidine binds to the remaining coordination sites as shown in  figure 2.2\cite{h6tag}. The minimal concentration of imidazole in wash buffer removes non specifically bound molecules in the column. Then,  elution buffer is passed through the column and elution was collected as it contains the partially pure protein. Higher concentration of imidazole is used as eluting agent as imidazole competes with Histidine tag for binding.
   

    
\begin{wrapfigure}{t}{0.5\textwidth}
    \includegraphics[width=0.5\textwidth]{pics/niNTA.png}
    \caption{Complex formed by poly-His tag, Ni$^{2+}$ and NTA \cite{niNTA_pic}}
    \label{fig:enter-label}
\end{wrapfigure} \\
\begin{table}
\centering   
\begin{tabular}{cc}
\begin{tabularx}{5.7cm}{ 
  | >{\centering\arraybackslash}X 
  | >{\centering\arraybackslash}X  | }
 \hline
 Tris & 50 mM  \\
 \hline
 NaCl  & 300 mM    \\
\hline
 EDTA  & 20 mM    \\
 \hline
 \multicolumn{2}{|c|}{pH=8, using HCl}    \\
 \hline
 \multicolumn{2}{|c|}{Filter with 0.45\textmu\ filter paper}  \\
 \hline
 \end{tabularx}
&
 \begin{tabularx}{5.7cm}{ 
  | >{\centering\arraybackslash}X 
  | >{\centering\arraybackslash}X  | }
 \hline
 Tris & 50 mM  \\
 \hline
 NaCl  & 300 mM    \\
 \hline
 EDTA  & 250 mM    \\
 \hline
\multicolumn{2}{|c|}{pH=8 , using HCl}  \\  
\hline
\multicolumn{2}{|c|}{Filter with 0.45\textmu\ filter paper}  \\
\hline
\end{tabularx}
\label{table:ta5}
\end{tabular}
\caption{Wash and Elution buffer}
    \label{tab:my_label}
\end{table}\\




\paragraph{Procedure}First, NTA-Agarose bed  was prepared. The column was then connected to the output tube of a  peristaltic pump. Before connecting the tubes of the pump to the column and buffer, the tubes were purged to remove air bubbles inside the tubes. Purging was done by immersing the input tube in a solvent and running the pump in a higher flow rate. During purge, output tube shouldn't be connected to the column. After purging, column was charged by passing 100 mM NiSO\textsubscript{4}.6H\textsubscript{2}O through the column by immersing the input tube in the respective solvent and connecting the output tube to the Ni-NTA column. Volume of each buffer passed was equivalent to 2/3 times the column volume and column volume is 20 mL. The flow rate is 1 mL/minute. The bed will turn light blue from white due to $Ni^{2+}$ ions. Next, the column was equilibrated with lysis buffer, as the pellet was suspended in lysis buffer before cell lysis. The post cell lysis supernatant is passed through the column and the flow-through is collected. Then, wash \& elution buffer were passed and their respective flow-through is collected. Ni-NTA column was washed by first removing the $Ni^{2+}$ from the coulmn, by passing 100 mM EDTA through the column. EDTA and NTA ligands competes to bind with $Ni^{2+}$ and as EDTA is a hexadendate ligand, it is more successful in binding. Then, MilliQ is passed . The wash is complete. The bed is kept intact by closing the bottom of the column making sure that the solvent is prsent in the column. We can use the same column again by repeating the cycle. To make sure that the affinity chromatography went well, the flow-through after running sample, wash flow-through and elution buffer would be injected in SDS PAGE. If there concentrated dark band for elution flow-through, then affinity chromatography is successful.

\subsection{SDS - PAGE}\\
\begin{table}[t]
\begin{tabular}{cc}
\begin{tabularx}{7cm}{ 
  | >{\centering\arraybackslash}X 
  | >{\centering\arraybackslash}X  | }
  \hline
  \multicolumn{2}{|c|}{\textbf{Sealing Gel}}    \\
\hline
 Acrylamide & 500 \textmu L \\
 \hline
 APS(10\%)  & 10 \textmu L    \\
 \hline
 TEMED  & 2 \textmu L    \\
 \hline
\end{tabularx}&

\begin{tabularx}{7cm}{ 
  | >{\centering\arraybackslash}X 
  | >{\centering\arraybackslash}X  | }
 \hline
 \multicolumn{2}{|c|}{\textbf{TGS Buffer}}    \\
\hline
 Tris - Base & 3.04 g \\
 \hline
 SDS & 1 g \\
 \hline
 Glycine & 14.42 g    \\
 \hline
 Water & Make up the volume to 1 L \\
 \hline
 \multicolumn{2}{|c|}{pH = 8.3 using conc HCl}    \\
\hline
 \end{tabularx}\\

\begin{tabularx}{7cm}{ 
  | >{\centering\arraybackslash}X 
  | >{\centering\arraybackslash}X  | }
 \hline
 \multicolumn{2}{|c|}{\textbf{1X Staining Dye}}    \\
\hline
 Glycerol & 1 mL \\
 \hline
 Tris HCl(1M,pH=6.8) & 0.6 mL \\
 \hline
 SDS(10\%)  & 2 mL    \\
 \hline
 Water  & 5.1 mL    \\
 \hline
 $\beta$-marceptoethanol  & 0.5 mL    \\
 \hline
 \multicolumn{2}{|c|}{pinch of Bromophenol Blue}    \\
\hline
\end{tabularx}&

 \begin{tabularx}{7cm}{ 
  | >{\centering\arraybackslash}X 
  | >{\centering\arraybackslash}X  | }
 \hline
 \multicolumn{2}{|c|}{\textbf{Resolving Gel}}    \\
\hline
 Acrylamide & 1.875 mL \\
 \hline
 Water & 1.825 mL \\
 \hline
 Tris HCl (1.5M,pH=8.8) & 0.975 mL \\
 \hline
 SDS(10\%)  & 37.5 \textmu L    \\
 \hline
 APS(10\%)  & 37.5 \textmu L    \\
 \hline
 TEMED  & 1.5 \textmu L    \\
 \hline
\end{tabularx}\\


\begin{tabularx}{7cm}{ 
  | >{\centering\arraybackslash}X 
  | >{\centering\arraybackslash}X  | }
 \hline
 \multicolumn{2}{|c|}{\textbf{Stainer(100 mL)}}    \\
\hline
 Coomassie Brilliant Blue & 0.25 g \\
 \hline
 Water & 45 mL \\
 \hline
 Methanol & 45 mL \\
 \hline
 Glacial acetic acid  & 10 \textmu L    \\
 \hline
 \end{tabularx}&


 \begin{tabularx}{7cm}{ 
  | >{\centering\arraybackslash}X 
  | >{\centering\arraybackslash}X  | }
 \hline
 \multicolumn{2}{|c|}{\textbf{Stacking Gel}}    \\
\hline
 Acrylamide & 247.5 \textmu L \\
 \hline
 Water & 1.05 mL \\
 \hline
 Tris HCl (1M,pH=6.8) & 187.5 \textmu L \\
 \hline
 SDS(10\%)  & 15 \textmu L    \\
 \hline
 APS(10\%)  & 15 \textmu L    \\
 \hline
 TEMED  & 1.5 \textmu L    \\
 \hline
\end{tabularx}
\label{table:ta8}\\


\begin{tabularx}{7cm}{ 
  | >{\centering\arraybackslash}X 
  | >{\centering\arraybackslash}X  | }
 \hline
 \multicolumn{2}{|c|}{\textbf{De-Stainer(100 mL)}}    \\
\hline
 Water & 45 mL \\
 \hline
 Methanol & 45 mL \\
 \hline
 Glacial acetic acid  & 10 \textmu L    \\
 \hline
 \end{tabularx}\\
 \end{tabular}
\caption{\textbf{SDS PAGE Recipe}}
\label{tab:my_label}
\end{table}\\

Sodium Dodecyl Sulphate - Polyacrylamide Gel Electrophoresis is a protein separation method by mass. First, SDS PAGE apparatus is setup and sealing gel was made to seal the bottom part. 10\% resolving gel and stacking gel were prepared. Sample's tertiary structure is degraded by addition of 10\textmu L of Staining dye to 10 \textmu L sample(flow-through, wash and elution), heated in boiling water for 10 mins, centrifuged at 6000 rpm for 3 minutes. 15 \textmu L was loaded into each of well, apparatus was immersed in 1X TGS Buffer and run at 120 V until the stainer in the sample reaches the end of the gel. Then, gel was taken out from the apparatus, immersed in stainer for 2 - 3 hours, rinsed with water and was immersed in de-stainer for 1 hour.





\subsection{Dialysis}\\

This method is based on diffusion due to solvent concentration difference. The concentration of imidazole had to be eliminated as much as possible as it acts as an inhibitor against thrombin. This was done by pooling the eluted solution in SnakeSkin Dialysis Tubing, and placing the bag in the dialysis solution. The buffer is replaced for every 4 - 6 hours, and this is done for 4 hours, so approximately 24 hours of dialysis. Now, the sample is ready for thrombin cleavage.

\subsection{Thrombin Cleavage}

\begin{wraptable}{H}{5cm}

\begin{tabularx}{5cm}{ 
  | >{\centering\arraybackslash}X 
  | >{\centering\arraybackslash}X  | }
  \hline
 Tris & 50 mM  \\
 \hline
 NaCl  & 200 mM    \\
\hline
\multicolumn{2}{|c|}{pH=7, using HCl}    \\
\hline
\end{tabularx}
\caption{Dialysis buffer}
\label{table:ta5}
\end{wraptable} 
 The expressed fusion protein has thrombin cleavage site [ L V P R \big\downarrow\ G S ]right after the sequence of cytochrome sequence in order to cleave out the peptide of interest from the his tag and cytochrome. 5 mM of CaCl$_2.$2H$_2$O was added to the elution and the thrombin reaction was carried out in the ratio of protein:Thrombin = 1:25 at 20\degree C for 16 hours.\cite{Thrombin}
 

\subsection{Size Exclusion Chromatography}
\begin{wraptable}{H}{6cm}

\begin{tabularx}{5.7cm}{ 
  | >{\centering\arraybackslash}X 
  | >{\centering\arraybackslash}X  | }
\hline
 Tris & 50 mM  \\
 \hline
 NaCl  & 200 mM    \\
\hline
Sodium azide & 0.01\%    \\
\hline
\multicolumn{2}{|c|}{pH=7, using HCl} \\  
\hline
\multicolumn{2}{|c|}{Filter with 0.2 \textmu\ filter paper}    \\
\hline
\end{tabularx}
\caption{SEC Buffer}
\label{table:ta5}

\end{wraptable}

 \paragraph{Principle}
 The compound is segregated according to its size/Stokes' radius/hydrodynamic volume. The stationary phase is made of inherently hydrophobic, chemically and physically inert polystyrene/divinylbenzene copolymers so that they do not interfere with the analyte. Counterintuitively, the bigger molecules elute out first and the smaller ones elute later. The column has "hollow beads" through which small molecules can pass, but the bigger molecules can't. The travel path for bigger molecules is shorter than that of smaller molecules. So, smaller the molecules, later they elute. The elution is passed through a detector which gives absorption spectra at different wavelengths. Different spectral peaks correspond to different peptides. The fraction corresponding to the spectral peak has our peptide of interest. Separation parameters depend on the SEC column used.
 \paragraph{Procedure}
Seperation of peptide from impurities(importantly Cytb\textsubscript{5} and His tag) was achieved using Superdex-30 column. The column was equilibriated with 1 column volume of SEC Buffer. 2 mL of sample was injected in the column and elution was collected in 2ml fraction at flow rate of 1 mL/min.

 \subsection{Concentrating using Amicon}
\paragraph{Principle} Amicon centrifugal tubes have a molecular weight cutoff filter separating the tube into 2 parts. The sample is loaded in the top part and centrifuged. The filter retains all the compounds which is higher than the molecular weight cutoff and rest is enters the bottom part as centrifuge.

\paragraph{Procedure}
The SEC fraction corresponding to the peptide peak was loaded into Amicon Ultra Centrifugal Filter 3 kDa MWCO and centrifuged in 3500 rpm, 4 \degree C for 30 minutes. The steps were repeated until the whole of SEC fractions were over and until concentrated sample was obtained

 \subsection{Reverse Phase HPLC}
 \paragraph{Principle}The purest form of the peptide is obtained through this step. This is done by passing the sample through a column whose matrix(stationary phase) and mobile phase(solvent+sample) interacts eith the sample eluting different compounds at different times due to the strength of interaction. Contrary to normal HPLC, stationary phase is more nonpolar than the eluting solvent in reverse phase HPLC. Main components are nonpolar stationary phase, e.g., C \textsubscript{18} silica  and a moderately polar aqueous mobile phase. One common RP-HPLC stationary phase is surface-modified silica, RM\textsubscript{2}SiCl, where R is a straight chain alkyl group such as C\textsubscript{8}H\textsubscript{17} or C\textsubscript{18}H\textsubscript{37}. The peptides bind to the stationary matrix through hydrophobic interactions and gets eluted due to the interaction with mobile phase. Types of flow of mobile phase are isocratic and gradient flow based on the concentration of solvents throughout the run. So, elution occurs when the hydrophobic interaction between peptide and stationary phase is weaker than interaction between mobile phase and peptide due specific composition of solvents at that particular time. Thus, different peptides elute out at different times according to the strength of the hydrophobic retention. The elution is passed through a UV/Vis detector , which gives a spectra. Each peak in the HPLC chromatogram correspond to a certain peptide and is collected. The peak that appears in the very start corresponds to the molecules which didn't bind to the column, and is termed as the void peak.

 \paragraph{Procedure}
 The concentrated sample was run in Agilent 1200 HPLC machine. 100 \textmu L of sample was injected into LiChroCART 18e,300\r{A},5\textmu\   column, with (H\textsubscript{2}O+0.1\% TFA) and (ACN+0.1\%TFA) as solvents applied as 35 minutes linear gradient and detected at 280 nm. Sample which eluted during appearance of distinct were collected and analysed using MALDI Mass spectrometry to verify the efficiency of expression and purification.\\

\begin{figure}[b]
    \centering
    \includegraphics[width=\textwidth]{pics/hplc.jpg}
    \caption{Basic workflow of HPLC}
    \label{fig:enter-label}
\end{figure}
 



 \subsection{Lyophilisation}
 Also known as freeze drying, lyophilisation is used to isolate the peptide from volatile compounds(in this case, HPLC solvents) by sublimation. This is done by first cooling down the elution, results in formation of ice crystals. Now, the pressure is decreased and temperature is increased slowly. As volume and pressure are indirectly proportional, the ice tends to expand and the volatile substances like the HPLC solvents sublimate. The vapour is collected and condensed. This is done in 3 steps 
 
  Pre Freezing - Sample cooled to temperatures below it's freezing point, forming homogenous ice crysals
  
Primary Drying - Temperature is increased slowly, a considerable amount of volatile substance is sublimated

Secondary Drying - Pressure is decreased and temperature is increased, sublimating the volatile compounds completely

The cycle is repeated for a few times. The sample which was suspended in solvents comes out as powder after lypohylisation.

\begin{figure}[H]
    \centering
    \includegraphics[width=0.5\linewidth]{pics/lyoph.jpg}
    \caption{Lyophiliser machine}
    \label{fig:enter-label}
\end{figure}





\chapter{Results}
\section{Transformation}
    
\begin{figure}[H]

\begin{subfigure}{0.3\linewidth}
\includegraphics[width=0.95\linewidth, height=4cm]{pics/transformed/bmk1.jpg} 
\caption{Bmk M1}
\label{fig:subim1}
\end{subfigure}
\begin{subfigure}{0.27\linewidth}
\includegraphics[width=0.95\linewidth, height=4cm]{pics/transformed/lqq4.jpg}
\caption{Lqq4}
\label{fig:subim2}
\end{subfigure}
\begin{subfigure}{0.27\linewidth}
\includegraphics[width=0.95\linewidth, height=4cm]{pics/transformed/lqtxy13s.jpg} 
\caption{Lqh$\alpha$IT mutant}
\label{fig:subim3}
\end{subfigure}

\caption{Transformed cells after overnight incubation in 37\degree C}
\label{fig:image}
\end{figure}

\section{Affinity Chromatography}
\begin{figure}[H]
\begin{subfigure}{0.3\linewidth}
\includegraphics[width=0.95\linewidth, height=6cm]{pics/sds/bmk1.png} 
\caption{Bmk M1}
\label{fig:subim1}
\end{subfigure}
\begin{subfigure}{0.3\linewidth}
\includegraphics[width=0.95\linewidth, height=6cm]{pics/sds/lqq4.png}
\caption{Lqq4}
\label{fig:subim2}
\end{subfigure}
\begin{subfigure}{0.3\linewidth}
\includegraphics[width=0.95\linewidth, height=6cm]{pics/sds/lqtx.PNG} 
\caption{Lqh$\alpha$IT mutant}
\label{fig:subim3}
\end{subfigure}

\caption{SDS-PAGE of flow-through, wash and elution of respective peptides in 10\% Resolving gel}
\label{fig:image}
\end{figure}

\section{Size Exclusion Chromatography}

\begin{figure}[H]
\begin{subfigure}{1\linewidth}
\includegraphics[width=1\linewidth, height=6cm]{pics/sec/bmk1.jpeg} 
\caption{Bmk M1}
\label{fig:subim1}
\end{subfigure}

\begin{subfigure}{1\linewidth}
\includegraphics[width=1\linewidth, height=6cm]{pics/sec/lqq4_N15.jpeg}
\caption{Lqq4}
\label{fig:subim2}
\end{subfigure}

\begin{subfigure}{1\linewidth}
\includegraphics[width=1\linewidth, height=6cm]{pics/sec/lqtx_dlc.jpeg} 
\caption{Lqh$\alpha$IT mutant}
\label{fig:subim3}
\end{subfigure}
\caption{SEC Chromatograms of respective peptides}
\label{fig:image}
\end{figure}
Pink region represents the Elution Volume collected

\section{HPLC}

\begin{figure}[H]
\begin{subfigure}{1\linewidth}
\includegraphics[width=1\linewidth, height=6cm]{pics/hplc/bmk1.jpeg} 
\caption{Bmk M1}
\label{fig:subim1}
\end{subfigure}

\begin{subfigure}{1\linewidth}
\includegraphics[width=1\linewidth, height=6cm]{pics/hplc/lqq4_N15.jpeg}
\caption{Lqq4}
\label{fig:subim2}
\end{subfigure}

\begin{subfigure}{1\linewidth}
\includegraphics[width=1\linewidth, height=6cm]{pics/hplc/lqtxy13s_dl.jpeg} 
\caption{Lqh$\alpha$IT mutant}
\label{fig:subim3}
\end{subfigure}
\caption{HPLC Chromatograms of respective peptides }
\label{fig:image}
\end{figure}
Yellow region represents Solvent A(H\textsubscript{2}O+0.1\%TFA).\\
Blue region represents Solvent B(ACN+0.1\%TFA).

\section{Mass Spectrometry}
\begin{figure}[H]
    \centering
    \includegraphics[width=1\linewidth]{pics/mass.png}
    \caption{MALDI MS of post HPLC unlabelled Lqq4}
    \label{fig:enter-label}
\end{figure}
Expected mass: 7469.53 Da
\newline
Mass from MALDI: 7470.46 Da
\newline
MALDI clearly shows that unlabelled Lqq4 was successfully purified\\

Transformation was successful as good number of colonies have grown on the ampicillin LB agar plate.\\

Distinct bands in elution column in SDS Page suggests that affinity chromatography was successful in partially purifying the peptide.\\

Distinct peak in SEC and RP-HPLC spectra suggests that the peptides were well purified

\chapter{Conclusion}
This summer internship project successfully detailed the methodologies for the expression and purification of three significant scorpion toxins: BmK M1, Lqq4, and the Lqh$\alpha$IT mutant (Y11S). The comprehensive approach, encompassing molecular cloning, bacterial expression, and multi-stage protein purification, provided hands-on experience in advanced protein biochemistry techniques.

The project commenced with the successful transformation of E. coli BL21(DE3) cells with pET21a(+) plasmids containing the respective toxin genes. Subsequent IPTG induction facilitated the expression of these toxins as fusion proteins. Cell lysis through the French press method effectively released the intracellular proteins, preparing them for downstream processing.

The purification involved a sequential application of established chromatographic techniques. Ni-NTA affinity chromatography proved effective in isolating the His-tagged fusion proteins, as evidenced by SDS-PAGE analysis showing concentrated bands in the elution fractions and minimal protein in the flow-through and wash steps. Following affinity purification, thrombin cleavage successfully liberated the target peptides from their fusion partners. Size exclusion chromatography (SEC) further refined the purification by separating the peptides effectively removing larger contaminants such as the cleaved Cytb$_{5}$ and His-tag. Finally, reverse-phase HPLC (RP-HPLC) provided the high-resolution separation necessary to achieve the desired purity, with distinct peaks corresponding to the target peptides. The successful purification of unlabelled Lqq4 was confirmed by MALDI mass spectrometry, which showed a mass consistent with the expected molecular weight.

This internship has been a profoundly formative experience, providing a deep understanding of practical protein biochemistry methodologies, expression system optimization, and the biochemical behavior of disulfide-rich peptides. The challenges encountered during the various stages of purification, such as optimizing buffer conditions and troubleshooting chromatographic runs, offered critical problem-solving opportunities. The insights gained into the expression and purification of these specific scorpion toxins, particularly their unique structural features like the non-proline cis peptide bond, contribute to a broader understanding of their functional roles and potential therapeutic applications. This experience has built a strong foundation for future research endeavors in protein science and structural biology.

\chapter{Future Scope}
\begin{itemize}

\item Verification of successful purification of remaining peptides using MALDI MS

\item Expression and purification of isotope labelled peptides

\item Analysing the unlabelled and labelled peptides using NMR
\end{itemize}


\printbibliography[heading=bibintoc, title={Bibliography}]
\end{document}
